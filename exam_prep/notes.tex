\documentclass{article}
\usepackage[utf8]{inputenc}
\usepackage{amsmath}
\usepackage{amssymb}
\usepackage{color}
\usepackage{graphicx}
\usepackage{float}
\usepackage{physics}
\title{notes.tex }
\author{Henrik Linder}
\date{\today}
\begin{document}
\maketitle

\begin{center}
\begin{tabular}{ |l|c| } 
 \hline
 &\\
 Cell volume &1$\mu $m$^3$ \\ 
  Length of DNA&1550 $\mu $m \\ 
  Length of DNA &$5\cdot 10^{6}$ bp \\ 
  Average protein size&360 AAs \\ 
					  Av. diameter of protein&5 nm\\
					  Conc. of proteins&5-8mM\\
					  Diameter of ribosome&20 nm\\
					  Volume fraction of proteins&17\%\\
					  Volume fraction of all RNA&6\%\\
					  Volume fraction of mRNA&.2\%\\
					  Volume fraction of DNA&1\%\\
					  Volume fraction of ribosomes&8\%\\
					  Number of mRNA molecules&4 000\\
					  Number of rRNA molecules&18 000\\
					  Number of tRNA molecules&200 000\\
					  Number of all proteins&3 500 000\\
					  Number of amino acids&$10^9$\\
					  Number of free amino acids&6 000 000\\
					  Number of ribosomes&40 000\\
					  &\\
					  Half-life mRNA& 5 min\\
					  Half-life E.Coli& 30 min\\
					  Conc. 1 protein in cell&1 nM\\
					  Translation rate / ribosome&13.8 $\frac{\text{codons}}{\text{ribosome}\cdot s}$\\
					  Translation rate / ribosome&41.4 $\frac{\text{bps}}{\text{ribosome}\cdot s}$\\
					  &\\
					  &\\
					  &\\
 \hline
\end{tabular}
\end{center}

Diffusion time across a distace $d$ in the E.Coli cell: 
\begin{equation}
	t_{\text{diffusion}} = \frac{d^2}{2D}
\end{equation}
where $D$ is diffusion const. (Typically of the order .1-10 $\mu m^2/s$)

On-time is the time it takes for diffusing particle to find its target in the volume $V$
\begin{equation}
	\frac{1}{\tau_{on}} = \frac{4\pi D\epsilon N}{V}
\end{equation}
I think $N$ is number of particles needed for turning on. 

Searching along DNA, we get 
\begin{equation}
	t_{on} = \frac{L^2}{D}
\end{equation}


\end{document}
