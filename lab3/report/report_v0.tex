\documentclass{article}
\usepackage[utf8]{inputenc}
\usepackage{amsmath}
\usepackage{amssymb}
\usepackage{color}
\usepackage{graphicx}
\usepackage{float}
\usepackage{physics}
\usepackage{subcaption}
\title{report v0.tex }
\author{Henrik Linder}
\date{\today}
\begin{document}
\maketitle



\section{Task 1: Determining overall structure}
\subsection{Theory}
\subsection{Results and discussion}
\begin{figure}[H]
	\centering
	\begin{subfigure}[b]{.3\textwidth}
		\centering
		\includegraphics[width=\linewidth]{"figs/task1/degree_dist_v1.eps"}
		\caption{}
		\label{fig:task1_degree_dist}
	\end{subfigure}
	\begin{subfigure}[b]{.3\textwidth}
		\centering
		\includegraphics[width=\linewidth]{"figs/task1/clustering_coeff_v1.eps"}
		\caption{}
		\label{fig:task1_clustering_coeff}
	\end{subfigure}
	\begin{subfigure}[b]{.3\textwidth}
		\centering
		\includegraphics[width=\linewidth]{"figs/task1/spl_v1.eps"}
		\caption{}
		\label{fig:task1_spl}
	\end{subfigure}
\end{figure}

\section{Task 2: Comparison against random networks}
\subsection{Theory}
The Erdös-Réyni (ER) random network follows a simple algorithm. 
\begin{enumerate}
	\item Create a network with $N$ nodes 
	\item For each node $n_i$, connect $n_i$ to $n_j\neq n_i$ with a probability $p$.
\end{enumerate}

Given this generation algorithm, the average number of degrees (connections to another node) for any one node is 
\begin{equation}
	\begin{split}
	\langle k\rangle &= (\text{\#Possible connections })\cdot(\text{ Prob. of connection})\\
						 &=\left( \sum_{n_i\neq n_j}^N p \right)\\
						 &= (N-1)p.
	\end{split}
\end{equation}

\subsection{Results and discussion}

\begin{figure}[H]
	\centering
	\begin{subfigure}[b]{.3\textwidth}
		\centering
		\includegraphics[width=\linewidth]{"figs/task2/human_kidney_protein_clustering_coeff_w_30_rand_v3.eps"}
		\caption{}
		\label{fig:task2_clust_hkp}
	\end{subfigure}
	\begin{subfigure}[b]{.3\textwidth}
		\centering
		\includegraphics[width=\linewidth]{"figs/task2/yeast_gene_network_clustering_coeff_w_30_rand_v5.eps"}
		\caption{}
		\label{fig:task2_clust_yeast}
	\end{subfigure}
	\begin{subfigure}[b]{.3\textwidth}
		\centering
		\includegraphics[width=\linewidth]{"figs/task2/e.coli_metabolic_net_clustering_coeff_w_30_rand_v1.eps"}
		\caption{}
		\label{fig:task2_clust_ecoli}
	\end{subfigure}
	\begin{subfigure}[b]{.3\textwidth}
		\centering
		\includegraphics[width=\linewidth]{"figs/task2/human_kidney_protein_spl_w_30_rand_v1.eps"}
		\caption{}
		\label{fig:task2_spl_hkp}
	\end{subfigure}
	\begin{subfigure}[b]{.3\textwidth}
		\centering
		\includegraphics[width=\linewidth]{"figs/task2/yeast_gene_network_spl_w_rand_v3.eps"}
		\caption{}
		\label{fig:task2_spl_yeast}
	\end{subfigure}
	\begin{subfigure}[b]{.3\textwidth}
		\centering
		\includegraphics[width=\linewidth]{"figs/task2/e.coli_metabolic_net_spl_w_30_rand_v1.eps"}
		\caption{}
		\label{fig:task2_spl_ecoli}
	\end{subfigure}
\end{figure}



\section{Task 3: Robustness in biology}
\subsection{Theory}
The diameter of a network is defined as the longest of all shortest paths in the network. 
\subsection{Results and discussion}

\begin{figure}[H]
	\centering
	\begin{subfigure}[b]{.49\textwidth}
		\centering
		\includegraphics[width=\linewidth]{"figs/task3/e.coli_metabolic_net_random_rm_20_runs_per_f_v0.eps"}
		\caption{Randomly removing nodes.}
		\label{fig:task3_rand}
	\end{subfigure}
	\begin{subfigure}[b]{.49\textwidth}
		\centering
		\includegraphics[width=\linewidth]{"figs/task3/e.coli_metabolic_net_top_rm_20_runs_per_f_v1.eps"}
		\caption{Removing top nodes.}
		\label{fig:task3_top}
	\end{subfigure}
\end{figure}


\end{document}
