\documentclass{article}
\usepackage[utf8]{inputenc}
\usepackage{amsmath}
\usepackage{amssymb}
\usepackage{color}
\usepackage{graphicx}
\usepackage{float}
\usepackage{physics}
\title{lec7.tex }
\author{Henrik Linder}
\date{\today}
\begin{document}
\maketitle


\section{Repetition}
Probability of binding in a certain volume is 
\begin{equation}
	P_b = \frac{1}{1 + e ^ {\beta\Delta G}} = \frac{1}{1 + \frac{1}{V}e ^ {-\beta\epsilon}}
\end{equation}
depending on how you define your energy. 


$\Delta G = $ free energy = energy + entropy

Huge entropy cost of folding DNA.

Non-specific binding means that $P_b = 1/2$ 


Corporate binding : counteract the entropy term in the binding prob by having two binding sites next to eachother. 
\begin{equation}
	\Delta G  = \epsilon + \epsilon ' - k_b T \ln(\Omega) + \epsilon _{R - R'}
\end{equation}
where $k_b T \ln(\Omega)$ is the entropy. $\Omega$ is the \# of boxes = V/a$^3$.



\section{Stochastic dynamics in living systems}
\begin{itemize}
	\item Diffusion
	\item Target search
\end{itemize}

Examples: 

- Protein diffusion: 
	Molecule moving around in a cell. 

	If you want to move in a straight line, you will have to spend energy. 

	Random thermanl motion cossts nothing. 

- Target search :
	

- Transcription initiation: 
	


\subsection{Diffusion}
Incredibly low Reynolds numbers in cells. 

Check out "Life at low Reaynolds numbers" again. 



3D position of the particle: 
\begin{equation}
	\vec r = (x,y,z)
\end{equation}
%fig 6 

Equation of motion: patricle in a viscuous fluid. 

N2: 
\begin{equation}
	m \dv[2]{x}{t} = \text{drag force + noise (+ external force)}
\end{equation}
Drag force $\propto$ velocity at low vel.s.
\begin{equation}
	m \dv[2]{x}{t} = \xi \dv{x}{t} = \eta (t)
\end{equation}
where $\xi$ is the hydrodynamic friction: $\xi =$(6$\pi$viscosity/radius)

in the high-friction limit: \begin{equation}
	\dv[2]{x}{t} = 0
\end{equation}

\begin{equation}
	\dv {x}{t} = \frac{1}{\xi}\eta(t)
\end{equation}
which is the Lengevin eq. (Brownian motion)

This is an unbiased random walk. 
(also used in stock market analysis (geometric brownian motion))


What is $\eta(t)$? White noise. Random noise with 


1) mean = 0.
%fig 7

many types of averages. Ensemble average: "n statistical mechanics, the ensemble average is defined as the mean of a quantity that is a function of the microstate of a system, according to the distribution of the system on its micro-states in this ensemble."
\begin{equation}
	<\eta(t)> = 0
\end{equation}

2) variance over time is constant. 

%fig 8

3) uncorrelated

if you correlate two different regions, the correlation should be 0. Corr. between $T_1$ and $T_1$ = 1, but  between $T_1$ and $T_n$ = 0 for $n \neq 1$ 

White noise is the least biased noise you can add. 
\begin{equation}
	<\eta(t)\eta(t')>\leftarrow(t-t')
\end{equation}

calculate displacement: 

Integrate Langevin eq. ($\int_0^t\dd u$)
\begin{equation}
	x(t) - x(0) = \Delta x = \frac{1}{\xi}\int _0^t\eta (u)\dd u
\end{equation}
\begin{equation}
	<\Delta x> = <\frac{1}{\xi}\int_0^t<\eta(u)>\dd u = 0
\end{equation}

-Calc mean-squared displacement (MSD):

\begin{equation}
	\begin{split}
	<\Delta x (t)^2> = <\Delta x(t)\Delta x(t)> \\
	&= <\frac{1}{\xi}\int_0^t\eta(u)\dd u \cdot \int_0^t \eta (v)\dd v\\
	&= \frac{1}{\xi^2} \int_0^t \int_0^t \text{const} \cdot\delta (u-v)\dd u \dd v \\
	&= \frac{C}{\xi^2} \int_0^t \dd u \\
	&= \frac{Ct}{\xi^2}
	\end{split}
\end{equation}
which is $\equiv 2Dt$ where $D$ is the diffusion constant. 

special for diffusion: square of distance traveled is prop- to time, dist $\propto \sqrt t$, much slower than ballistic motion. 


All directions are independent: 
\begin{equation}
	<\Delta x^2>
	=<\Delta y^2>
	=<\Delta z^2>
\end{equation}

gives that 
\begin{equation}
	\begin{split}
	<\Delta R^2> = <\Delta \vec R \cdot \Delta \vec R > = <\Delta x^2 = \Delta y ^2 + \Delta z ^ 2> = 3 <\Delta x ^ 2> = 6 \Delta t
	\end{split}
\end{equation}

How fast wiill RNAp diffuse across E.Coli? 

Diameter 1 $\mu m $

$D = 5(\mu  m ^2)/s$

$<\Delta R^2 = 6Dt>$
 \begin{equation}
	t = \frac{<\Delta R^2>}{6D} = 0.03
\end{equation}
	seconds. 

	Pretty fast. 

What about human cells? 

Diameter 10 $\mu m $

$D = 5(\mu  m ^2)/s$

$<\Delta R^2 = 6Dt>$

\begin{equation}
	t = 3
\end{equation}
seconds. 


How long to diffuse along E.Coli's DNA? 

1D diffusion. 

$l_{DNA} = 1550 \mu m$  (calc from \# BPs and size. )

$D= \frac{5 (\mu m )^2}{s}$

\begin{equation}
	t = \frac{l_{DNA}^2}{2D} = \frac{2.4e6(\mu m )^2}{2\cdot 5 \frac{(\mu m )^2}{s}} \approx 67 hrs \approx 3 \text{days}
\end{equation}


\subsection{What is the distribution of x?}
P(x,t)

x is the sum of random numbers, so it will conform to a gaussion for large numbers. 

Will tend to a gaussian curve with mean $M = 0$, $\sigma^2 = 2Dt$ 
\begin{equation}
	\implies P(x,t) = \frac{e ^ {-(x-x_0)^2/4Dt}}{\sqrt{4\pi Dt}}
\end{equation}
$P(x,t)$ satisfieas 

\begin{equation}
	\pdv {D}{t} = D\pdv[2] {P}{x}
\end{equation}

independent directions, so 

\begin{equation}
	\begin{split}
	P(\vec r , t ) = p(x,y,z,t) = p(x,t)p(y,t)p(z,t)\\
	= \frac{e^{-x^2/4Dt}}{\sqrt{4\pi Dt}}\cdot
	 \frac{e^{-y^2/4Dt}}{\sqrt{4\pi Dt}}\cdot
	 \frac{e^{-z^2/4Dt}}{\sqrt{4\pi Dt}}\\
	 = \frac{e^{-(x^2 + y^2 + z^2)}}{(4\pi Dt)^{3/2}}\\
	\end{split}
\end{equation}


\subsection{Target search}
Find the search time $t$ from volume, D, a. 


\begin{equation}
	t \approx \frac{V}{aD}
\end{equation}
\begin{equation}
	t = \frac{V}{4\pi aD}
\end{equation}



\end{document}
