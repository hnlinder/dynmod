\documentclass{article}
\usepackage[utf8]{inputenc}
\usepackage{amsmath}
\usepackage{amssymb}
\usepackage{color}
\usepackage{graphicx}
\usepackage{float}
\usepackage{physics}
\title{lec4.tex }
\author{Henrik Linder}
\date{\today}
\begin{document}
\maketitle


\section{Lab info}
Translation is the topic of the lab 

\section{Recap}
Binding probability is sigmoid curve as a function of R (concentration)

Prob 1/2 is at R = k, with k = $\frac{R}{R + k}$ = concentration$\approx 1e-4 \quad 1e-2 \mu M$


To steepen the curve, 

\begin{equation}
	\frac{R ^ 4}{R ^ 4 + k_2}
\end{equation}



\section{Translation}
tRNA with an amino acid and a kodon matches a kodon on the mRNA with th ehelp of the ribosome which then chains the amino acids together into a protein. 


Protein is sometiemes called a polypeptide since it's a chain of AAs with peptide bonds. 

\begin{equation}
	DNA \longrightarrow [transcription] \longrightarrow mRNA \longrightarrow [translation] \longrightarrow proteins
\end{equation}

1 codon is one specific amino acid 
ex: 


codon combinations are $4^3 = 64$, there are $\approx$ 20 AAs, so we have a degenerate code. 

so if the tRNA codon is uac , then the mRNA is AUG (complementary codon, anti-codon)

There's a translation speed that's on average 12.5 codons/s

1 protein is 360 AAs is 360 codons. Translation time 360/12.5 $\approx$ 20 s

as an exercise : calculate on average how many ribosomes there are per RNA
(exc 1000 mRNA, 40000 ribosomes, so 40 ribosomes per RNA. Hence traffic jams.)

\section{Traffic jam model}
\subsection{Lab 1: 1D traffic model for translation}
1D lattice representing the mRNA. with lattice sites (boxes) representing codons. 
ribisomes moving = particles jumping forward with rate $q$.
if there's another particles in the box ahead, the one behind cannot jump. 
to enter the system with entrance rate $\alpha$ and exit the system with exit rate $\beta$.

goal i to calculate the particle flux through the system, from i = 1 to i = L (and exit). 

$\theta$ = Prob that i is occupied by a ribosome. 

%i $\equiv$2,..,L-1
\begin{equation}
	\begin{split}
	\dv{\theta_i}{t} = q\theta_{i-1}(i-\theta_i) -q\theta_i(i-\theta_{i+1}), i \equiv 2,..,L-1\\
	\dv{\theta_1}{t} = \alpha(1-\theta_1) -q\theta_1(1-\theta_{2})\\
	\dv{\theta_L}{t} = \theta_{L-1}(1-\theta_L) - \beta \theta_L\\
	\end{split}
\end{equation}

Flux $J= \beta \theta_L$



Two solutions  : 
1)
\begin{equation}
	\theta_1 = \theta_i = 1,\quad\theta_L=0,\quad J = o
\end{equation}
2)
\begin{equation}
	\theta_1 = \theta_i = \frac{\alpha(1-\alpha)}{\beta},\quad J = \alpha(1-\alpha)
\end{equation}

noit correct.

an implicit assumtion in the derivation of $J$, is that $\beta >> \alpha$ is large so that it is diluted. 





\section{Lab info}
Translation is the topic of the lab 

Follow the report guidelines, short but correct
make subplots and combine plots. 


lab1 

genereal problem: when you have multiple ribosomes, you get traffic jams, 

for the later parts: different codons have different translation rates. 


for the "stochastic" part of the lab, just calculate the probabilities and check if the event happens. 

in every lattice point, its state can be $\tau_i =  \{1,2\}$ (free or occupied). What complicates things is that $\tau_i$ and $\tau_{i\pm 1}$ are correlated. 

\begin{equation}
	<\tau_i(t)\tau_{i-1}>\approx <\tau_i(t)><\tau_{i-1}(t)> \equiv \theta_i(t)\theta_{i-1}(t)
\end{equation}
which is the mean-field approximation, valid in the low-density regime. 

the high-beta low density thing is a good sanity check in the lab


\section{transcription regulation}
\subsection{Protein degrsdation }
\begin{equation}
	\dv{c}{t}  = prod(R) - ck_{deg}
\end{equation}

the degradation term is different for activators and suppressors. 

activator: 
\begin{equation}
	prod(R)\propto \frac{R}{R+ K}
\end{equation}
Degradation has two components. 

1) dilution by cell division. CD doubles the volume, but keeps the cumber of proteins the same. Dilution time scale is roughly cell life time. In bacteria, roughly 1/2 hours. RNA lifetime, roughly minutes. 

2) active degradation. An enzyme binds to the protein and chops it up back into AAs.
\begin{equation}
	k_{deg} = \frac{1}{\tau_{cell}} + k_{deg,a}
\end{equation}

If proteins misfold and form aggregrates (which are problematic in cells), there are special pacman protein sthat cut them apart, that's the active degradation part. 

If you turn off protein production, how fast will $c(t) \longrightarrow 0$, $c$ is protein concentration. 
\begin{equation}
	\begin{split}
	\dv {c}{t} = P_0 \frac{R}{R+ k} - k_{deg}c\\
	p_0 = k_{deg} \bar c
	\end{split}
\end{equation}

proteins degrade as 
\begin{equation}
	e ^ {-k_{deg}t}
\end{equation}
Response time (to go to zero) 
\begin{equation}
	t_{resp}\approx \frac{1}{k_{deg}} \approx \tau_{cell}
\end{equation}
this is way too slow. 
cells must regulate genes in other ways to acchieve fast response timesl 

this is called seimple regulation 

ex: self-regulation (auto-regulation)
(negative feedback loop)




\end{document}
