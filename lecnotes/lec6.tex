\documentclass{article}
\usepackage[utf8]{inputenc}
\usepackage{amsmath}
\usepackage{amssymb}
\usepackage{color}
\usepackage{graphicx}
\usepackage{float}
\usepackage{physics}
\title{lec6.tex }
\author{Henrik Linder}
\date{\today}
\begin{document}
\maketitle



\section{Repetition}
smallRNA regulation 

sRNA binds to mRNA, stopping translation 

much faster than transcription regulation. 

\begin{equation}
	\dv {c}{t} = prod(R) - k_{deg}c
\end{equation}


The responsiveness of this system is determined by the removal rate $\gamma$

by responsiveness i mean the slope of the concentration of sRNA. 

\begin{equation}
	\begin{split}
	\dv {m}{t} = \alpha_m = \gamma m s - m/\tau_m\\
	\dv {s}{t} = \alpha_s = \gamma m s - s/\tau_s\\
	\end{split}
\end{equation}
might be important fo exam to be able to formulate these kinds of equations. 




\begin{equation}
	\bar m \approx \frac{1}{(\alpha -1)\gamma}
\end{equation}


\section{Statistical mechanichs of binding}
%fig 1 

Probability of binding: 
\begin{equation}
	P_b = \frac{R}{R+k}
\end{equation}
R is concentration, k is energy??

A balance between energy and entropy. 

We consider the opposite limit, one particle (protein), many sites. 

%fig 2 

$P_k$ is prob to be in site k 
$P_i = \frac{1}{\Omega}$ 

\begin{equation}
	\sum _{i = 1}^{\Omega}P_k = \frac{1}{\Omega}\sum _{i = 1}^{\Omega} = 1 
\end{equation}
makes sense! 
%fig 3
\begin{equation}
	P_k = \frac{1}{z}e ^ {-\beta \epsilon_b},\quad P_i = \frac{1}{z}e ^ {-\beta \epsilon_0}
\end{equation}

\begin{equation}
	\begin{split}
	1 = \sum _{i = 1}^\Omega  
	\end{split}
\end{equation}


partition function is 
\begin{equation}
	Z = e ^{-\beta \epsilon_b} + (\Omega - 1)e^{-z\beta\epsilon_0}
\end{equation}


binding prob (density): 
\begin{equation}
	\begin{split}
	P_b = \frac{e ^ {-\beta\epsilon_b} }{e ^ {-\beta\epsilon_b} + \Omega e ^ {-\beta\epsilon_0} } \\
	= \frac{(1/\Omega )e ^ {-\beta(\epsilon_b - \epsilon_0} }{(1/\Omega )e ^ {-\beta(\epsilon_b - \epsilon_0}+1}\\
	= \frac{1}{1 + e ^ {\beta\Delta G} }\\
	\end{split}
\end{equation}
since 
\begin{equation}
	\begin{split}
	e ^ {-\beta(\epsilon_b - \epsilon_0) + \ln (1/\Omega )} =  e ^ {-\beta(\epsilon_b - \epsilon_0) + \ln (1/\Omega )}
	\end{split}
\end{equation}


what about several binding sites? L sites
 

\begin{equation}
	\begin{split}
		P_b = \frac{(L/\Omega)e ^ {-\beta\Delta\epsilon} }{(L/\Omega)e ^ {-\beta\Delta\epsilon} + 1}\\
			\implies \Delta G = \Delta \epsilon + k_bT\ln(L/\Omega)
	\end{split}
\end{equation}


\section{Non-specific binding to DNA}

The transcription factors (TFs) have a binding affinity to many places on DNA, not just the right ones.


\begin{equation}
	P_b = 1/2
\end{equation}
at what $\Delta \epsilon$?

Assume $R + 1,\Omega \gg L, L\gg 1$



\begin{equation}
	P_b = \frac{1}{1 + e^{\beta \Delta G}}
\end{equation}

$e ^ {\beta\Delta G}  = 1$  

In ecoli 

V = 1 $\mu m ^3$, $\alpha  = 1nm$ (protein radius)

$l_{DNA} =$ 5e6 BPs $\implies L = 5e6$ BPs.



\begin{equation}
	\Delta \epsilon  = -k_bT \ln(\frac{5\cdot 10^6}{1-^9}) = -3\ln5 k_bT \approx -4.8 k_bT
\end{equation}
suspicious minus sign, check it. 

just convention apparently

this is a reasonable binding energy though. 

in humans: 

V = 10 $\mu m ^3$, $l_{DNA} = 7e10 $ bp , 

\begin{equation}
	\Delta \epsilon = k_bT \ln(\frac{L}{\Omega}) = k_bT \ln \frac{7e9}{10e10} = -1.9k_bT
\end{equation}

much lower binding energy 

therefore you will have much more binding. 

the length of the DNA matters a lot for the amount of binding. 




Check out  statistical mechanics of binding at bio-physics.at. 




Many regulators, one binding site:
showing \begin{equation}
	\frac{R}{R+k}
\end{equation}
\begin{equation}
	p_b = \frac{(R/\Omega) e ^ {\beta\Delta G}}{(R/\Omega e ^ {\beta\Delta G} + 1)}
\end{equation}

\begin{equation}
	\frac{R}{\Omega} = \frac{R}{V}\frac{V}{R} = \frac{[R]}{\Omega /V}
\end{equation}


\begin{equation}
	P_b = \frac{[R]}{[R] + \frac{V}{\Omega}e ^ {\beta\Delta G} }
\end{equation}
and defining 
\begin{equation}
	\frac{V}{\Omega}e ^ {\beta\Delta G}  = k
\end{equation}

k  is a property if the system, "standard state": $C_0 = .6M \approx 1M$. 

you always have to make sure that you're in the right standard state. 

\begin{equation}
	k \approx 10^{-8\pm 2}M
\end{equation}
in cell biology. 




\subsection{Corporate binding}
means that we have a binding site close to another BS. Then they have some binding energy $\Delta G$ between them. Standard energy is the same, but then you add corporate binding energy, which counteract teh entropy term, making it more probable to bind. 

Cooperativity, read in the book. This made a lot of genes occur in pairs. 


\subsection{DNA-looping}
there's another contribution to the binding energy from the DNA looping. 

\begin{equation}
	\Delta G_{loop} = T\Delta S_{loop}= T\ln (\text{prob to for a loop of size }l)
\end{equation}


simplest possible polymer is random orientation. looping prob is what are the odds of ending up where you started?

for rand polymer: 
\begin{equation}
	P(l) \approx \left(\frac{l}{l_0}\right)^{-3/2}
\end{equation}

%measuring the distance from 
measuring this in DNA will get you 
there are experiments measuring this in DNA

how many contacts did you get for 100,1000,10000 apart? 

in human: 
\begin{equation}
	P(l) \approx \left(\frac{l}{l_0}\right)^{-1.08}
\end{equation}

In the lac-repressor, we looked at 2 operator sites, there's lots of looping involved. most comlicated is the octamere, exptremely stable. 












\end{document}
