\documentclass{article}
\usepackage[utf8]{inputenc}
\usepackage{amsmath}
\usepackage{amssymb}
\usepackage{color}
\usepackage{graphicx}
\usepackage{float}
\usepackage{physics}
\title{lec5.tex }
\author{Henrik Linder}
\date{\today}
\begin{document}
\maketitle


\section{Repetition}
Translation: going from mRNA to proteins. 

Triplets of BPs called codons translate to AAs. Degenerate system, 64 codons, about 20 AAs. 

about 40 000 ribosomes, about 1000 RNAs (check  tab;e on p23)

Proteins degrade, usually too slow, needs active degradation. 

mRNA is messenger RNA, its a protein recipe. 

tRNA holds an anti-codon, is caught by ribisome and translated to aminoacid.
rRNA is ribosomal RNA, binds together in the ribosome ???

sRNA is small RNA, regulates translation. 

human DNA is about 7e9 BPs, 20000 genes (much longer genes than in bact), in bacteria 5e6 BPs, 4500 genes, much higher gene density. 

about 2\% productive DNA in humans, the rest is "junk DNA", "dark matter of the genom", lots of RNA comes from there. 

inhumans, long noncoding RNA. can be involved in folding and combining RNAs and proteins, mey be an important part of the expression of genes. 

RNA can do much more than DNA, so some think that RNA came first. DNA is much more stable though. 

\section{Small RNA regulation}
Post-translational regulation by small RNA. 

Gene ->[transcription]-> mRNA ->[translation]-> protein

Also called translation regulation (post-tanslation regulation). Shuts down the 

The mechanism is that there are small bits of RNA, that just binds to matching pieces of the mRNA and "blocks" the ribosome cannot pass, effectively stopping the translation. 

Since the half-life of the RNA is so short, the entire complex ends up degrading wiwhtout being translated. 


Example: Iron uptake ($Fe^{2+}$)

move $Fe^{2+} $ from the outside of the cellmembrane to the inside through ion channels. used in proteins containing iron. 
there's an issue/tradeoff: high concentrations of $Fe^{2+} $ are toxic. so the cell wants to keep the $Fe^{2+} $ concentration low but stable.

so the iron binds to a transcription factor (TF) called Fur. Has active and inactive form Fur and Fur*. Fur is a repressor that binds to an operator site. Binds to a gene called RhyB that codes for sRNA. 

Fur* ---$|$ RhyB (inhibits.)
RhyB inhibits prod of protein containing iron. 


if $Fe^{2+} $ conc lowers, Fur increases, iron use lowers. 


\section{Exercises}
\subsection{Q 2.3.2}
On average, 360 AA/protein. How many proteins are there in the cell? 

\#AAs $\approx$ 1e9
(\#free AAs $\approx$ 6e6, the vast majority are tied up in proteins. )

\#proteins = 1e9/360 $\approx$3e6

(remember:

typical size of protein $\approx 360$ AAs.

1/2 s to move across cell)

\subsection{Q 3.1.1}
Binding probs of R. 
approximation :  $\frac{R}{R+K}$ is valid for $\#R>> \gg\#$ bindings sites.

we should hace that 
\begin{equation}
	[OR] = \frac{[O][R]}{k + [R]}
\end{equation}
see plot in notebook. These curves will differ from eachother. As Ot $\longrightarrow$ 0, the blue line will approach the oragne one. 
\end{document}
