\documentclass{article}
\usepackage[utf8]{inputenc}
\usepackage{amsmath}
\usepackage{amssymb}
\usepackage{color}
\usepackage{graphicx}
\usepackage{float}
\usepackage{physics}
\title{lec12.tex }
\author{Henrik Linder}
\date{\today}
\begin{document}
\maketitle


chapters 1-4 in uri alon's book "An introduction to systems biology. "

\section{last lecture}
--Feed forward loops (FFL)

x promotes y promotes z and x promotes z

x -> y -> z \& x->z

(called a delay element)

-- GRNs (gene regulatory networks)

S1, !2, ....., Sn

the layer separating the Signals from the genes (about 4500) is the transcirtion factors. 

%fig 1 




most common regulatory motif: auto-regulation, x -> x or x -| x

pro's

shorter respons times 

good for homeostasis (can keep itself on a reasonable level )

\begin{equation}
	\dv{x}{t} = \beta f(x) - \alpha x
\end{equation}
\begin{equation}
	x(t_{1/2}) = \frac{\bar x}{2}
\end{equation}
simple reg. : $t_{1/2}= \frac{\ln 2}{\alpha}$ 

self repression : 
\begin{equation}
	f(x) = \left(\frac{k}{x}\right) ^2
\end{equation}
$t_{1/2} = [h = 1] = 0.2 \frac{\ln2}{\alpha}$

you should know this for exam: calc half times and response times




\section{FFL}
x->y->z and x->z
%fig 2
%fig 3


\begin{equation}
	y(t) = \bar y (1-e^{-\alpha t})
\end{equation}
where 
\begin{equation}
	\bar y = \frac{\beta_y}{\alpha _y}
\end{equation}
add or subtract $Y_0$ in some way to get the appropriate behavior as $t\leftarrow \infty$. 



prob of binding is associated with a binding const. $K_{yz}$. In order to have enough binding of $y*$, we must hace a concentration above a threshold level. this causes a delay in the production of z. the same is not true for x, we assume the productino of $x*$ is instantaneous. 

effective response time for z is now response time plus delay, so $T_{ON} + \frac{\ln2}{\alpha}$

we are assuming z behaves as an AND gate here. 

%fig 4

now y is repressing z: 

x->y-|z and x->z

z starts growing immediately and then backs down. Called pulse. 
%fig 5


easy to calc delay time $T_{ON}$. 

\begin{equation}
	y*(t = T{ON}) = k_{yz}
\end{equation}
gives 
\begin{equation}
	T{ON} = \frac{1}{\alpha _y}\ln\left(\frac{1}{1-\frac{k_yz}{\bar y}}\right)
\end{equation}
size of $k_yz$ is micromolar (??)
$\frac{k_yz}{\bar y}\approx 1/3 \cdot 1/10$

Exam q: given a GRN, how does x affect z or what is the response time of z.







\section{Feedback loops, FBL}
\section{Types of regulatory links}
pretty obvious in the previous example. 

Two main link types. The first is basically the previous: 

i) binding to promotors. ex :
%fig 6

ii) small molecules binding to proteins and protein-protein binding. ex:
%fig 7


example of ii), the lac-operon. 
%fig 8 
%fig 9 

when there's no lactose: lecrep blocks transcription of lac Z/Y/A. 

lactose present: => allo-lactose present => lacrep falls off 

this R-E-S system in fig 9 is a very common regulatory motif, e.g. metabolic regulation, stress-response systems. 


\subsection{Model R-E-S system}

%fig 10 


\begin{equation}
	\begin{split}
	\dv {E}{t} = \beta _E f(R)- \alpha_EE\\
	\dv {S}{t} = \beta _S- \gamma_{ES}ES
	\end{split}
\end{equation}
wherre R is "free R", not bound, 
\begin{equation}
	\frac{R_{free}}{k_{ER} + R_{free}}
\end{equation}
R + S forms a complex RS. with 

\begin{equation}
	\frac{RS}{R_{tot}} = \frac{k_{RS}}{k_{RS} + S}
\end{equation}

\begin{equation}
	R \frac{k_{RS}}{k_{RS} + S} = \#R \text{not bound by S }
\end{equation}





heat-shoch response: when heat goes up in cell, they produce chaperone proteins that untwist proteins to avoid protein aggregates. This follows a similar regulation network. 

common test for parkinsons : heat up e.coli, let cells divide and all the aggregates will be pushed to one side, collecting all aggregates in one cell . 





\end{document}
