\documentclass{article}
\usepackage[utf8]{inputenc}
\usepackage{amsmath}
\usepackage{amssymb}
\usepackage{color}
\usepackage{graphicx}
\usepackage{float}
\usepackage{physics}
\title{lucas labwalkthrough.tex }
\author{Henrik Linder}
\date{\today}
\begin{document}
\maketitle

\section{Stochastic sims}
\subsection{Fix timestep}
--timestep $\Delta t$is fixed
? what $\Delta t$ should be used? 

can be changed throughout the simulation. 

look at $P_i$

In lab 1: 
\begin{equation}
	k_p^{mrna} = \frac{1}{600}[s^{-1}]
\end{equation}
\begin{equation}
	k_d = \frac{1}{1800}[s^{-1}]
\end{equation}

\begin{equation}
	\begin{split}
	P_p^{mRNA} = k_p^{mRNA}\Delta t\\
	P_d = k_d\Delta t
	\end{split}
\end{equation}

two options: $P_i <= 1$ and $P_i << 1$

in the first case you will get the wrong results, the second will give slow results.

rule of thumb: 
\begin{equation}
	\max (P_i) \approx 50\%
\end{equation}

\subsection{Gillespie method}
chance of nothing happeing is 
\begin{equation}
	1 - r \frac{t}{n}
\end{equation}

read in the book or the paper by gillespie. 


in modsim: when population goes to infinity, your solution becomes exact but it is very unstable.



\section{Labs}
\subsection{Traffic model}
Protein production  by translation

The tRNA will bind and unbind several times very quickly, decreasing the error rate

the process uses 25 times more energy than it should, compared to a computer where that number is 100s of thousands. 

algo: 

-pick $\Delta t$
-build a list if all events
-shuffle
-go through


task3 : go to t=10 000. expect, staggering

J is about .5 


task4 looks about right, and the histogram doesnt really matter

lower the timestep!!

\subsection{AUM model}
$\Delta t = 1$

$\alpha$: prob of active conversion

\texttt{if rand() < $\alpha$: 
	n1 : random site
	n2 : radnom site
	if n1==M and n2==U or F: 
		n2 =m or U
	else: 
		n1 : rand
		move on AUM}

\begin{equation}
	F = \frac{\alpha }{1-\alpha}
\end{equation}
\begin{equation}
	\begin{split}
	(1-\alpha ) = \alpha <=>F - \alpha F=\alpha\\
	\alpha = \frac{F}{1+F}
	\end{split}
\end{equation}
either active or random!!!

If the active event doesn't happen, the inactive should happen!!




\end{document}
