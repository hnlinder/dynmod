\documentclass{article}
\usepackage[utf8]{inputenc}
\usepackage{amsmath}
\usepackage{amssymb}
\usepackage{color}
\usepackage{graphicx}
\usepackage{float}
\usepackage{physics}
\title{lucas_notes.tex }
\author{Henrik Linder}
\date{\today}
\begin{document}
\maketitle


Prob $\alpha$ that it does something. Only one nucleosome should do something per time step 

from p128 

for each step you do active or noisy 

l conversions one time step, instead do l time steps with randomly chosen.


you do the conversion with prob $\alpha$

series of M/A. 

What's the prob that any nucleosome become sunspecific? 

\begin{equation}
	\begin{split}
	P(n\leftarrow U) = P(pick M)\cdot P(pick A) \cdot \alpha\\
	= \frac{M}{N} \frac{A}{N}\alpha = \alpha m \alpha
	\end{split}
\end{equation}

want to see the histograms from fig 7.8, 131

not time step dependent, only two things happening: active or random conversion. 



from lab1: 

just discuss why not every mRNA is 100\% efficient. 
 
my j is probably low because i take j from t =0 


\end{document}
